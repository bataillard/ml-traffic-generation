% ================================================ %
% Headers and metadata
% ================================================ %

\documentclass[10pt,conference,compsocconf]{IEEEtran}

\usepackage{hyperref}
\usepackage{graphicx}	% For figure environment

\begin{document}
\title{%
    Discrete traffic data generation using ML methods \\
    \large \textit{CS-433 Machine learning -- Project 2}
}

\author{
  Luca Bataillard, Julian Blackwell, Changling Li\\
  \textit{School of Computer and Communication Sciences, EPFL}
}

\maketitle


% ================================================ %
% Abstract
% ================================================ %

\section*{Abstract}

This paper introduces a way to generate synthetic discrete traffic data for use in traffic
simulators. We present a model capable of taking historical traffic data from a given week and
predicting the discrete arrival times of vehicles in the same week, one year later. This is achieved 
using two phases. In the first phase, a recurrent neural network predicts the arrival rate for each 
hour in the week. In the second phase, a custom discretization model transforms our rates into new 
discrete arrival times. Our model outperforms our baseline in predicting arrival rates, and is 
...TODO in generating arrival times.


% ================================================ %
% Introduction
% ================================================ %

\section{Introduction}

Traffic simulators like SUMO \cite{SUMO2018} need vehicle data to perform modelling tasks. These
simulators are widely used in the fields of civil engineering and urban planning for modelling complex
traffic systems. They can provide city-wide traffic forecasts, allow for the evaluation of novel
traffic control systems, or can even model the effect of autonomous vehicle on traffic patterns. 

To simulate traffic flow, simulators will generate simulated vehicles on a given road based on a 
log of the speed and arrival time of vehicles on that road. Such real world data is obtained using 
a car sensor placed on or near the roadway. These sensors are not always online: they can break 
down or can be moved to be used elsewhere. They also cannot predict future traffic trends. This 
means that the logs sometimes have large gaps if using real-world data.

In this paper, we present a way to fill in the gaps in traffic logs and to predict future traffic 
logs up to a year in advance, using machine leaning models. Our model takes a week-long traffic
log and outputs a traffic log for the same week one year later. It can be combined to predict an 
entire year in parallel using data from the previous year. 

TODO phases + challenges + results


% ================================================ %
% Exploratory data analysis
% ================================================ %

\section{Exploratory data analysis}

\subsection{TODO}

\subsection{Year-on-year traffic increases}

\subsection{Zero weeks}


% ================================================ %
% Feature engineering
% ================================================ %

\section{Feature engineering}

\subsection{Re-sampling}

\subsection{HGV Driving restrictions}

\subsection{Time periodicty}

\subsection{Splitting and normalizing}

\subsection{Windowing}

\subsection{Zero weeks filtering}


% ================================================ %
% Model structure and selection
% ================================================ %

\section{Model structure and selection}

\subsection{Rate prediction}

\subsubsection{Baseline}

\subsubsection{Linear model}

\subsubsection{Neural Network}

\subsubsection{LSTM}

\subsection{Discrete event generation}


% ================================================ %
% Experimental results
% ================================================ %

\section{Experimental results}

\subsection{Rate prediction performance}

\subsection{Discrete event generation performance}

% ================================================ %
% Future work
% ================================================ %

\section{Future work}

\subsection{Not treating weeks as independent}

\subsection{TODO...}


% ================================================ %
% Conclusion
% ================================================ %

\section{Conclusion}


% ================================================ %
% Bibliography
% ================================================ %


\bibliographystyle{IEEEtran}
\bibliography{report}

\end{document}